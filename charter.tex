\documentclass[
11pt, % The default document font size, options: 10pt, 11pt, 12pt
%codirector, % Uncomment to add a codirector to the title page
]{charter} 

% El títulos de la memoria, se usa en la carátula y se puede usar el cualquier lugar del documento con el comando \ttitle
\titulo{Cadenas de procesamiento re-configurables en SoC/MPSoC } 

% Nombre del posgrado, se usa en la carátula y se puede usar el cualquier lugar del documento con el comando \degreename
\posgrado{Carrera de Especialización en Sistemas Embebidos} 
%\posgrado{Carrera de Especialización en Internet de las Cosas} 
%\posgrado{Carrera de Especialización en Intelegencia Artificial}
%\posgrado{Maestría en Sistemas Embebidos} 
%\posgrado{Maestría en Internet de las cosas}

% Tu nombre, se puede usar el cualquier lugar del documento con el comando \authorname
\tituloAutor{Ing. }
\autor{Carlos Jorge Maffrand} 

% El nombre del director y co-director, se puede usar el cualquier lugar del documento con el comando \supname y \cosupname y \pertesupname y \pertecosupname
\tituloDirector{Mtr. Ing. }
\director{Gonzalo Lavigna}
\pertenenciaDirector{FIUBA} 


% Nombre del cliente, quien va a aprobar los resultados del proyecto, se puede usar con el comando \clientename y \empclientename
\tituloCliente{Mtr. Ing. }
\cliente{Franco Alcaraz}
\empresaCliente{INVAP}

% Nombre y pertenencia de los jurados, se pueden usar el cualquier lugar del documento con el comando \jurunoname, \jurdosname y \jurtresname y \perteunoname, \pertedosname y \pertetresname.
\juradoUno{Nombre y Apellido (1)}
\pertenenciaJurUno{pertenencia (1)} 
\juradoDos{Nombre y Apellido (2)}
\pertenenciaJurDos{pertenencia (2)}
\juradoTres{Nombre y Apellido (3)}
\pertenenciaJurTres{pertenencia (3)}
 
\fechaINICIO{24 de junio de 2021}		%Fecha de inicio de la cursada de GdP \fechaInicioName
\fechaFINALPlan{24 de agosto de 2021} 	%Fecha de final de cursada de GdP
\fechaFINALTrabajo{24 de junio de 2022}	%Fecha de defensa pública del trabajo final


\begin{document}

\maketitle
\thispagestyle{empty}
\pagebreak

\thispagestyle{empty}
{\setlength{\parskip}{0pt}
\tableofcontents{}
}
\pagebreak


\section*{Registros de cambios}
\label{sec:registro}


\begin{table}[ht]
\label{tab:registro}
\centering
\begin{tabularx}{\linewidth}{@{}|c|X|c|@{}}
\hline
\rowcolor[HTML]{EBEBEB} 
Revisión & \multicolumn{1}{c|}{\cellcolor[HTML]{EBEBEB}Detalles de los cambios realizados} & Fecha      \\ \hline
0      & Creación del documento                                 &\fechaInicioName \\ \hline
1      & Se genera la primera versión de los incisos 1, 2, 3, 4 y 5.                 & 04 de julio de 2021 \\ \hline
2      & Se genera la primera versión de los incisos 6, 7 y 8. & 08 de julio de 2021 \\ \hline
%		  Se puede agregar algo más \newline
%		  En distintas líneas \newline
%		  Así                                                    & dd/mm/aaaa \\ \hline
%3      & Se completa hasta el punto 11 inclusive                & dd/mm/aaaa \\ \hline
%4      & Se completa el plan	                                 & dd/mm/aaaa \\ \hline
\end{tabularx}
\end{table}

\pagebreak

\section*{Acta de constitución del proyecto}
\label{sec:acta}

\begin{flushright}
Buenos Aires, \fechaInicioName
\end{flushright}

\vspace{2cm}

Por medio de la presente se acuerda con el Ing. \authorname\hspace{1px} que su Trabajo Final de la \degreename\hspace{1px} se titulará ``\ttitle'', consistirá en \textcolor{black}{implementar un sistema que permita seleccionar bloques de procesamiento almacenado en memoria, realizar re-configuración dinámica en lógica programable y ofrecer un entorno de pruebas para el filtro seleccionado. Este entorno de pruebas debe ser capaz de generar una onda de excitación para el filtro, adquirir los datos filtrados y generar reportes de pruebas. Este desarrollo tiene como finalidad evaluar el uso de bancos de filtros mediante re-configuración dinámica de lógica programable para aplicaciones satelitales y de radar}, y tendrá un presupuesto preliminar estimado de \textcolor{red}{600} hs de trabajo y \textcolor{red}{\$ 0}, con fecha de inicio \fechaInicioName\hspace{1px} y fecha de presentación pública \fechaFinalName.

Se adjunta a esta acta la planificación inicial.

\vfill

% Esta parte se construye sola con la información que hayan cargado en el preámbulo del documento y no debe modificarla
\begin{table}[ht]
\centering
\begin{tabular}{ccc}
\begin{tabular}[c]{@{}c@{}}Ariel Lutenberg \\ Director posgrado FIUBA\end{tabular} & \hspace{2cm} & \begin{tabular}[c]{@{}c@{}}\clientename \\ \empclientename \end{tabular} \vspace{2.5cm} \\ 
\multicolumn{3}{c}{\begin{tabular}[c]{@{}c@{}} \supname \\ Director del Trabajo Final\end{tabular}} \vspace{2.5cm} \\
%\begin{tabular}[c]{@{}c@{}}\jurunoname \\ Jurado del Trabajo Final\end{tabular}     &  & \begin{tabular}[c]{@{}c@{}}\jurdosname\\ Jurado del Trabajo Final\end{tabular}  \vspace{2.5cm}  \\
%\multicolumn{3}{c}{\begin{tabular}[c]{@{}c@{}} \jurtresname\\ Jurado del Trabajo Final\end{tabular}} \vspace{.5cm}                                                                     
\end{tabular}
\end{table}

\section{1. Descripción técnica-conceptual del proyecto a realizar}
\label{sec:descripcion}

En los desarrollos tecnológicos, tanto aeroespaciales como de radares, es muy frecuente la utilización de procesamiento digital de las señales en dispositivos de lógica programable. En consecuencia, dado los elevados costos de los lanzamientos, las misiones espaciales y los desarrollos de defensa es necesaria la utilización eficiente de los recursos. Por lo que, mediante la re-configuración parcial, se podría reducir el número de dispositivos de lógica programable utilizados en futuros desarrollos.

El sistema propuesto en este plan de trabajo permitirá hacer una evaluación del uso de técnicas de re-configuración dinámica parcial en dispositivos de lógica programable para aplicaciones espaciales y de defensa.

Las \ttitle permiten seleccionar filtros de procesamiento, previamente generados, de un banco de filtros almacenado en memoria, realizar la re-configuración dinámica y evaluar el funcionamiento del banco seleccionado. Como componente principal, el dispositivo a desarrollar cuenta con un System on Chip (SoC) o Multi Processor System on Chip (MPSoC).

En la Figura \ref{fig:diagBloques} se muestra un diagrama de bloques en el que se observan las funcionalidades principales de la aplicación a desarrollar. De la misma, se desprende que el objetivo de la aplicación es generar un entorno de software-HDL que permita la selección de bancos de filtros que re-configuran la lógica programable del SoC/MPSoC.

\begin{figure}[h]
\raggedleft
\includegraphics[width=1.05\textwidth]{./Figuras/DiagamaCPR_grande.pdf}
\caption{Diagrama en bloques}
\label{fig:diagBloques}
\end{figure}

El entorno de pruebas se debe implementar haciendo uso de un kit de desarrollo que contenga un SoC o MPSoC de serie 7 o superior de la marca Xilinx. Por lo que, se deben establecer los mecanismos de re-configuración mediante el entorno de desarrollo Vivado.

El dispositivo debe constar de los principales bloques constitutivos:
\begin{itemize}
	\item Interfaz de usuario (UI): debe proveer al operador un modo de configurar el filtro de procesamiento empleado en el dispositivo, disparar la ejecución de pruebas y obtener los archivos de reporte. Adicionalmente, el operador debe poder actualizar los archivos que conforman el banco de filtros con los que se puede configurar la lógica programable (PL).
	\item Sistema de procesamiento (PS): debe contar con un sistema operativo que permita la ejecución de diferentes aplicaciones. Las mismas tienen como funcionalidades principales: controlar los periféricos del procesador, realizar la re-configuración de la PL, generar y administrar los reportes de pruebas, administrar el banco de filtros, configurar las pruebas a realizar y gestionar la comunicación con la PL mediante el bus AXI.
	\item Lógica programable (PL): en la misma se debe disponer de una parte estática donde se instancia una lógica de generación de estímulos y adquisición de datos para la realización de las pruebas sobre los filtros. Por otro lado, en la PL debe existir una sección configurable donde se deben instanciar los bloques de procesamiento.
	\item Memoria estática (SD): espacio de memoria donde se se deben almacenar el banco de filtros, los reportes de las pruebas ejecutadas, archivos de configuración del sistema y los datos de las pruebas recientemente realizadas por el usuario.
	\item Memoria RAM: a ser utilizada para la generación de reportes.
\end{itemize}

 
\section{2. Identificación y análisis de los interesados}
\label{sec:interesados}

\begin{table}[ht]
\begin{center}
\begin{tabularx}{\linewidth}{@{}|l|X|X|l|@{}}
\hline
\rowcolor[HTML]{EBEBEB} 
Rol           & Nombre y Apellido & Organización 	& Puesto 	\\ \hline
%Auspiciante   &  - 				  & -              	& -        	\\ \hline
Cliente       & \clientename      &\empclientename	& -       	\\ \hline
%Impulsor      & - &\empclientename 	& -       	\\ \hline
Responsable   & \authorname       & FIUBA        	& Alumno 	\\ \hline
%Colaboradores & -                 & -             	& -       	\\ \hline
Orientador    & \supname	      & \pertesupname 	& Director Trabajo Final \\ \hline
%Equipo        & -			      & -            	& -       	\\ \hline
%Opositores    & -                 & -            	& -      	\\ \hline
Usuario final & Operarios de sistemas especiales                 & \empclientename	& -       	\\ \hline
\end{tabularx}
\caption{Identificación de los interesados}
\label{tab:interesados}
\end{center}
\end{table}

\begin{itemize}
	\item Cliente: \clientename, es un profesional altamente calificado en la industria aeroespacial y defensa.
	\item Orientador: \supname, es egresado de la Maestría de Sistemas Embebidos de la FIUBA, posee muchos años de experiencia desarrollando proyectos en la industria aeroespacial y de defensa. Probablemente esté viviendo en Europa, por lo que hay que tener en cuenta ese hecho cuando se decida contactarlo.
	%\item Usuario final: profesional que aún no está definido por \empclientename.
\end{itemize}

\section{3. Propósito del proyecto}
\label{sec:proposito}

El propósito de este proyecto es realizar una evaluación tecnológica de la re-configuración dinámica de dispositivos de lógica programable, con el fin de establecer criterios de uso para desarrollos aeroespaciales y de radares. Adicionalmente, el cliente desea adquirir la capacidad de desarrollar una aplicación con esa tecnología de modo de subir el Technology Readiness Level (TRL) que se posee dentro de \empclientename.

\section{4. Alcance del proyecto}
\label{sec:alcance}

El proyecto descripto en este plan de trabajo deberá brindar las siguientes funcionalidades:

\begin{itemize}
	\item Administrar un banco de filtros previamente generados en un sistema de archivos.
	\item Seleccionar el filtro que se planea utilizar.
	\item Realizar la re-configuración parcial de la lógica programable del SoC/MPSoC, implementando el filtro previamente seleccionado.
	\item Configurar el entorno de pruebas presente en la parte estática de la PL.
	\item Ejecutar pruebas configuradas en las cadenas de filtros elegidas.
	\item Adquirir los datos generados en las pruebas.
	\item Emitir reportes sobre las pruebas ejecutadas.
\end{itemize}

El presente proyecto no incluye las siguientes funcionalidades:
\begin{itemize}
	\item Generar archivos binarios de filtros para la configuración dinámica de la PL.
	\item Rutinas de verificación, validación y evaluación de performance de las pruebas realizadas en los filtros configurados en la PL.
	\item Evaluación de los tiempos de re-configuración.
	\item Almacenamiento de grandes volúmenes de datos.
\end{itemize}

\section{5. Supuestos del proyecto}
\label{sec:supuestos}

Se supone que el software correrá en una plataforma de hardware tipo System on Chip de la línea Xilinx Zynq (ARM+FPGA).

El responsable del proyecto dispone de una placa ZedBoard para la realización del desarrollo. Con el fin de que el banco de pruebas sea implementado en una tecnología más moderna, se debe evaluar si INVAP dispone de un hardware de desarrollo que contenga un MPSoC.

Se asume que se dispondrá de la plataforma de desarrollo durante el tiempo que dure el proyecto para poder depurar el diseño sobre el hardware.

\section{6. Requerimientos}
\label{sec:requerimientos}

\begin{consigna}{red}
Los requerimientos deben numerarse y de ser posible estar agruparlos por afinidad, por ejemplo:

\begin{enumerate}
	\item Requerimientos funcionales
		\begin{enumerate}
			\item El sistema debe...
			\item Tal componente debe...
			\item El usuario debe poder...
		\end{enumerate}
	\item Requerimientos de documentación
		\begin{enumerate}
			\item Requerimiento 1
			\item Requerimiento 2 (prioridad menor)
		\end{enumerate}
	\item Requerimiento de testing...
	\item Requerimientos de la interfaz...
	\item Requerimientos interoperabilidad...
	\item etc...
\end{enumerate}

Leyendo los requerimientos se debe poder interpretar cómo será el proyecto y su funcionalidad.

Indicar claramente cuál es la prioridad entre los distintos requerimientos y si hay requerimientos opcionales. 

No olvidarse de que los requerimientos incluyen a las regulaciones y normas vigentes!!!

Y al escribirlos seguir las siguientes reglas:
\begin{itemize}
	\item Ser breve y conciso (nadie lee cosas largas). 
	\item Ser específico: no dejar lugar a confusiones.
	\item Expresar los requerimientos en términos que sean cuantificables y medibles.
\end{itemize}

\end{consigna}

\section{7. Historias de usuarios (\textit{Product backlog})}
\label{sec:backlog}

\begin{consigna}{red}
Descripción: En esta sección se deben incluir las historias de usuarios y su ponderación (\textit{history points}). Recordar que las historias de usuarios son descripciones cortas y simples de una característica contada desde la perspectiva de la persona que desea la nueva capacidad, generalmente un usuario o cliente del sistema. La ponderación es un número entero que representa el tamaño de la historia comparada con otras historias de similar tipo.

El formato propuesto es: "como [rol] quiero [tal cosa] para [tal otra cosa]."

Se debe indicar explícitamente el criterio para calcular los \textit{story points} de cada historia
\end{consigna}

\section{8. Entregables principales del proyecto}
\label{sec:entregables}

\begin{consigna}{red}

Los entregables del proyecto son (ejemplo):

\begin{itemize}
	\item Manual de uso
	\item Diagrama de circuitos esquemáticos
	\item Código fuente del firmware
	\item Diagrama de instalación
	\item Informe final
	\item etc...
\end{itemize}

\end{consigna}

\section{9. Desglose del trabajo en tareas}
\label{sec:wbs}

\begin{consigna}{red}
El WBS debe tener relación directa o indirecta con los requerimientos.  Son todas las actividades que se harán en el proyecto para dar cumplimiento a los requerimientos. Se recomienda mostrar el WBS mediante una lista indexada:

\begin{enumerate}
\item Grupo de tareas 1
	\begin{enumerate}
	\item Tarea 1 (tantas hs)
	\item Tarea 2 (tantas hs)
	\item Tarea 3 (tantas hs)
	\end{enumerate}
\item Grupo de tareas 2
	\begin{enumerate}
	\item Tarea 1 (tantas hs)
	\item Tarea 2 (tantas hs)
	\item Tarea 3 (tantas hs)
	\end{enumerate}
\item Grupo de tareas 3
	\begin{enumerate}
	\item Tarea 1 (tantas hs)
	\item Tarea 2 (tantas hs)
	\item Tarea 3 (tantas hs)
	\item Tarea 4 (tantas hs)
	\item Tarea 5 (tantas hs)
	\end{enumerate}
\end{enumerate}

Cantidad total de horas: (tantas hs)

Se recomienda que no haya ninguna tarea que lleve más de 40 hs. 

\end{consigna}

\section{10. Diagrama de Activity On Node}
\label{sec:AoN}

\begin{consigna}{red}
Armar el AoN a partir del WBS definido en la etapa anterior. 

%La figura \ref{fig:AoN} fue elaborada con el paquete latex tikz y pueden consultar la siguiente referencia \textit{online}:

%\url{https://www.overleaf.com/learn/latex/LaTeX_Graphics_using_TikZ:_A_Tutorial_for_Beginners_(Part_3)\%E2\%80\%94Creating_Flowcharts}

\end{consigna}

\begin{figure}[htpb]
\centering 
\includegraphics[width=.8\textwidth]{./Figuras/AoN.png}
\caption{Diagrama en \textit{Activity on Node}}
\label{fig:AoN}
\end{figure}

Indicar claramente en qué unidades están expresados los tiempos.
De ser necesario indicar los caminos semicríticos y analizar sus tiempos mediante un cuadro.
Es recomendable usar colores y un cuadro indicativo describiendo qué representa cada color, como se muestra en el siguiente ejemplo:



\section{11. Diagrama de Gantt}
\label{sec:gantt}

\begin{consigna}{red}

Existen muchos programas y recursos \textit{online} para hacer diagramas de gantt, entre los cuales destacamos:

\begin{itemize}
\item Planner
\item GanttProject
\item Trello + \textit{plugins}. En el siguiente link hay un tutorial oficial: \\ \url{https://blog.trello.com/es/diagrama-de-gantt-de-un-proyecto}
\item Creately, herramienta online colaborativa. \\\url{https://creately.com/diagram/example/ieb3p3ml/LaTeX}
\item Se puede hacer en latex con el paquete \textit{pgfgantt}\\ \url{http://ctan.dcc.uchile.cl/graphics/pgf/contrib/pgfgantt/pgfgantt.pdf}
\end{itemize}

Pegar acá una captura de pantalla del diagrama de Gantt, cuidando que la letra sea suficientemente grande como para ser legible. 
Si el diagrama queda demasiado ancho, se puede pegar primero la ``tabla'' del Gantt y luego pegar la parte del diagrama de barras del diagrama de Gantt.

Configurar el software para que en la parte de la tabla muestre los códigos del EDT (WBS).\\
Configurar el software para que al lado de cada barra muestre el nombre de cada tarea.\\
Revisar que la fecha de finalización coincida con lo indicado en el Acta Constitutiva.

En la figura \ref{fig:gantt}, se muestra un ejemplo de diagrama de gantt realizado con el paquete de \textit{pgfgantt}. En la plantilla pueden ver el código que lo genera y usarlo de base para construir el propio.

\begin{figure}[htbp]
\begin{center}
\begin{ganttchart}{1}{12}
  \gantttitle{2020}{12} \\
  \gantttitlelist{1,...,12}{1} \\
  \ganttgroup{Group 1}{1}{7} \\
  \ganttbar{Task 1}{1}{2} \\
  \ganttlinkedbar{Task 2}{3}{7} \ganttnewline
  \ganttmilestone{Milestone o hito}{7} \ganttnewline
  \ganttbar{Final Task}{8}{12}
  \ganttlink{elem2}{elem3}
  \ganttlink{elem3}{elem4}
\end{ganttchart}
\end{center}
\caption{Diagrama de gantt de ejemplo}
\label{fig:gantt}
\end{figure}


\begin{landscape}
\begin{figure}[htpb]
\centering 
\includegraphics[height=.85\textheight]{./Figuras/Gantt-2.png}
\caption{Ejemplo de diagrama de Gantt rotado}
\label{fig:diagGantt}
\end{figure}

\end{landscape}

\end{consigna}


\section{12. Presupuesto detallado del proyecto}
\label{sec:presupuesto}

\begin{consigna}{red}
Si el proyecto es complejo entonces separarlo en partes:
\begin{itemize}
	\item Un total global, indicando el subtotal acumulado por cada una de las áreas.
	\item El desglose detallado del subtotal de cada una de las áreas.
\end{itemize}

IMPORTANTE: No olvidarse de considerar los COSTOS INDIRECTOS.

\end{consigna}

\begin{table}[htpb]
\centering
\begin{tabularx}{\linewidth}{@{}|X|c|r|r|@{}}
\hline
\rowcolor[HTML]{C0C0C0} 
\multicolumn{4}{|c|}{\cellcolor[HTML]{C0C0C0}COSTOS DIRECTOS} \\ \hline
\rowcolor[HTML]{C0C0C0} 
Descripción &
  \multicolumn{1}{c|}{\cellcolor[HTML]{C0C0C0}Cantidad} &
  \multicolumn{1}{c|}{\cellcolor[HTML]{C0C0C0}Valor unitario} &
  \multicolumn{1}{c|}{\cellcolor[HTML]{C0C0C0}Valor total} \\ \hline
 &
  \multicolumn{1}{c|}{} &
  \multicolumn{1}{c|}{} &
  \multicolumn{1}{c|}{} \\ \hline
 &
  \multicolumn{1}{c|}{} &
  \multicolumn{1}{c|}{} &
  \multicolumn{1}{c|}{} \\ \hline
\multicolumn{1}{|l|}{} &
   &
   &
   \\ \hline
\multicolumn{1}{|l|}{} &
   &
   &
   \\ \hline
\multicolumn{3}{|c|}{SUBTOTAL} &
  \multicolumn{1}{c|}{} \\ \hline
\rowcolor[HTML]{C0C0C0} 
\multicolumn{4}{|c|}{\cellcolor[HTML]{C0C0C0}COSTOS INDIRECTOS} \\ \hline
\rowcolor[HTML]{C0C0C0} 
Descripción &
  \multicolumn{1}{c|}{\cellcolor[HTML]{C0C0C0}Cantidad} &
  \multicolumn{1}{c|}{\cellcolor[HTML]{C0C0C0}Valor unitario} &
  \multicolumn{1}{c|}{\cellcolor[HTML]{C0C0C0}Valor total} \\ \hline
\multicolumn{1}{|l|}{} &
   &
   &
   \\ \hline
\multicolumn{1}{|l|}{} &
   &
   &
   \\ \hline
\multicolumn{1}{|l|}{} &
   &
   &
   \\ \hline
\multicolumn{3}{|c|}{SUBTOTAL} &
  \multicolumn{1}{c|}{} \\ \hline
\rowcolor[HTML]{C0C0C0}
\multicolumn{3}{|c|}{TOTAL} &
   \\ \hline
\end{tabularx}%
\end{table}


\section{13. Gestión de riesgos}
\label{sec:riesgos}

\begin{consigna}{red}
a) Identificación de los riesgos (al menos cinco) y estimación de sus consecuencias:
 
Riesgo 1: detallar el riesgo (riesgo es algo que si ocurre altera los planes previstos de forma negativa)
\begin{itemize}
	\item Severidad (S): mientras más severo, más alto es el número (usar números del 1 al 10).\\
	Justificar el motivo por el cual se asigna determinado número de severidad (S).
	\item Probabilidad de ocurrencia (O): mientras más probable, más alto es el número (usar del 1 al 10).\\
	Justificar el motivo por el cual se asigna determinado número de (O). 
\end{itemize}   

Riesgo 2:
\begin{itemize}
	\item Severidad (S): 
	\item Ocurrencia (O):
\end{itemize}

Riesgo 3:
\begin{itemize}
	\item Severidad (S): 
	\item Ocurrencia (O):
\end{itemize}


b) Tabla de gestión de riesgos:      (El RPN se calcula como RPN=SxO)

\begin{table}[htpb]
\centering
\begin{tabularx}{\linewidth}{@{}|X|c|c|c|c|c|c|@{}}
\hline
\rowcolor[HTML]{C0C0C0} 
Riesgo & S & O & RPN & S* & O* & RPN* \\ \hline
       &   &   &     &    &    &      \\ \hline
       &   &   &     &    &    &      \\ \hline
       &   &   &     &    &    &      \\ \hline
       &   &   &     &    &    &      \\ \hline
       &   &   &     &    &    &      \\ \hline
\end{tabularx}%
\end{table}

Criterio adoptado: 
Se tomarán medidas de mitigación en los riesgos cuyos números de RPN sean mayores a...

Nota: los valores marcados con (*) en la tabla corresponden luego de haber aplicado la mitigación.

c) Plan de mitigación de los riesgos que originalmente excedían el RPN máximo establecido:
 
Riesgo 1: plan de mitigación (si por el RPN fuera necesario elaborar un plan de mitigación).
  Nueva asignación de S y O, con su respectiva justificación:
  - Severidad (S): mientras más severo, más alto es el número (usar números del 1 al 10).
          Justificar el motivo por el cual se asigna determinado número de severidad (S).
  - Probabilidad de ocurrencia (O): mientras más probable, más alto es el número (usar del 1 al 10).
          Justificar el motivo por el cual se asigna determinado número de (O).

Riesgo 2: plan de mitigación (si por el RPN fuera necesario elaborar un plan de mitigación).
 
Riesgo 3: plan de mitigación (si por el RPN fuera necesario elaborar un plan de mitigación).

\end{consigna}


\section{14. Gestión de la calidad}
\label{sec:calidad}

\begin{consigna}{red}
Para cada uno de los requerimientos del proyecto indique:
\begin{itemize} 
\item Req \#1: copiar acá el requerimiento.

\begin{itemize}
	\item Verificación para confirmar si se cumplió con lo requerido antes de mostrar el sistema al cliente. Detallar 
	\item Validación con el cliente para confirmar que está de acuerdo en que se cumplió con lo requerido. Detallar  
\end{itemize}

\end{itemize}

Tener en cuenta que en este contexto se pueden mencionar simulaciones, cálculos, revisión de hojas de datos, consulta con expertos, mediciones, etc.  Las acciones de verificación suelen considerar al entregable como ``caja blanca'', es decir se conoce en profundidad su funcionamiento interno.  En cambio, las acciones de validación suelen considerar al entregable como ``caja negra'', es decir, que no se conocen los detalles de su funcionamiento interno.

\end{consigna}

\section{15. Procesos de cierre}    
\label{sec:cierre}

\begin{consigna}{red}
Establecer las pautas de trabajo para realizar una reunión final de evaluación del proyecto, tal que contemple las siguientes actividades:

\begin{itemize}
	\item Pautas de trabajo que se seguirán para analizar si se respetó el Plan de Proyecto original:
	 - Indicar quién se ocupará de hacer esto y cuál será el procedimiento a aplicar. 
	\item Identificación de las técnicas y procedimientos útiles e inútiles que se emplearon, y los problemas que surgieron y cómo se solucionaron:
	 - Indicar quién se ocupará de hacer esto y cuál será el procedimiento para dejar registro.
	\item Indicar quién organizará el acto de agradecimiento a todos los interesados, y en especial al equipo de trabajo y colaboradores:
	  - Indicar esto y quién financiará los gastos correspondientes.
\end{itemize}

\end{consigna}


\end{document}
